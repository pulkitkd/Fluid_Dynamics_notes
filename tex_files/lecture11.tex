\documentclass[11pt,a4paper]{article}
\usepackage[utf8]{inputenc}
\usepackage{amsmath}
\usepackage{amsfonts}
\usepackage{amssymb}
\usepackage{graphicx}
%\usepackage[ddmmyyyy]{datetime} 
\usepackage[short,nodayofweek,level,12hr]{datetime} 
%\usepackage{cite}
%\usepackage{wrapfig}
%\usepackage[left=2cm,right=2cm,top=2cm,bottom=2cm]{geometry}

\newcommand{\e}{\epsilon}
\newcommand{\dl}{\delta}
\newcommand{\pd}[2]{\frac{\partial #1}{\partial #2}}
\newcommand{\vect}[1]{\underline{#1}}
\newcommand{\uvect}[1]{\hat{#1}}
\newcommand{\1}{\vect{1}}
\newcommand{\grad}{\nabla}


\title{Lecture 11: Self Gravitating Equilibria}
\date{\displaydate{date}}
\newdate{date}{10}{09}{2018}
\author{}

\begin{document}
\maketitle
\section*{Overview}

\section{Self-gravitating equilibrium for a spherical planet}
The gravitational potential $\Psi_G$ due to a small mass $dm$ is
\begin{align*}
d\Psi_G &= -\frac{G dm}{|{\vect x - \vect x_0}|}
\end{align*}
And hence for a body of volume $V$, the potential is 
\begin{align*}
\Psi_G &= -\int_V \frac{G \rho_0 dV}{|{\vect x - \vect x_0}|}
\end{align*}
For the potential in the exterior of a spherical body, this integral is trivial
\begin{align*}
\Psi_G &= -\int_V \frac{G \rho_0 dV}{|{\vect x - \vect x_0}|}\\
&=-\frac{G \rho_0}{r} \int_V dV \\
&=-\frac{G M}{r} 
\end{align*}
This gives us the familiar result that potential (and field) at an external point can be calculated by assuming the entire mass $M$ to be concentrated at the center of the body.

In order to calculate the potential in the interior region of a spherical body, we must perform the integral $\int_V \frac{dV}{r} = \int_V \frac{dV}{|\vect x - \vect x_0|}$ in spherical coordinates. The result we get is
\begin{align*}
\Psi_G &= -2 \pi \rho_0 G (R^2 - \frac{r^2}{3})
\end{align*}
At $r=R$ both interior and exterior regions give the same potential, so the result seems consistent. 

Since potential in the interior region is quadratic in $r$, and force is gradient of the potential, the force will be linear in $r$. Since $r = |\vect x - \vect x_0|$, it implies that the force is only a function of $\vect x$ (since the center $\vect x_0$ is fixed). The force is given by
\begin{align*}
F_G = -\frac{4\pi}{3}\rho_0 G\vect x
\end{align*}

This is the integral approach for obtaining the force due to a gravitating sphere. There is also a differential approach that utilises the idea of Green's function of a Laplacian. Recall that the Green's Function of a Laplacian in 3D is given by $G(\vect x) = -\frac{1}{4\pi|\vect x -\vect x_0|}$. Using the equation of the potenial, taking the Laplacian and using the Green's Function, we can obtain the same result as follows:
\begin{align*}
&\Psi_G = -\int_V \frac{G \rho_0 dV}{|{\vect x - \vect x_0}|}\\
&\grad^2 \Psi_G = 4\pi G \rho_0 \int_V \grad^2(\frac{-1}{4\pi|{\vect x - \vect x_0}|}) dV\\
&\grad^2 \Psi_G = 4\pi G \rho_0 \int_V \delta(\vect x - \vect x_0) dV \tag{\footnotesize{by defintion of Green's Function}}\\
&\grad^2 \Psi_G = 4\pi G \rho_0  \tag{\footnotesize{by defintion of Delta Function}}\\
&\frac{1}{r^2}\frac{d}{dr} r^2 \frac{d\Psi_G}{dr} = 4\pi G \rho_0 \\
&r^2 \frac{d\Psi_G}{dr} =\frac{4\pi G \rho_0 r^3}{3} \\
&\frac{d\Psi_G}{dr} =\frac{4\pi G \rho_0 r}{3} \\
&-\vect F =\frac{4\pi}{3}G \rho_0 r \1_r \\
&-\vect F =\frac{4\pi}{3}G \rho_0 (\vect x - \vect x_0) \\
\end{align*}
This is a radial(directed towards the center) harmonic force. The harmonic nature can be easily shown
\begin{align*}
\grad^2 \Psi_G &= 4\pi G \rho_0 \\
\grad (\grad^2 \Psi_G) &= 0  \tag{\footnotesize{taking $\grad$ on both sides}} \\
\grad^2 \grad \Psi_G &= 0  \tag{\footnotesize{$\grad$ and $\grad^2$ commute}}\\
\grad^2 \vect F &= 0  \tag{\footnotesize{$F = \grad \Psi_G$ }}
\end{align*}
To summarize,
\begin{itemize}
\item Interior: $\vect F \propto r \quad V \propto r^2$
\item Exterior: $\vect F \propto 1/r^2 \quad V \propto 1/r$
\end{itemize}

The fact that force has a linear dependence on $\vect x$ can also be seen using the fact that $\vect F$ is harmonic. Since $\grad^2 \vect F = 0$, it implies that $\vect F$ can be represented using the solution of the Laplacian. We know that if $\grad^2 \Phi = 0$, then
\begin{align*}
\Phi &= \underbrace{F(r)}_{\propto r^n, \frac{1}{r^{n+1}}}\quad \underbrace{G(\cos(\theta))}_{P^m_n(\cos(\theta))}\quad \underbrace{H(\phi)}_{e^{\pm i m \phi}}
\end{align*}
where $n$ is the order of tensor involved. The solution constructed using $r^n$ are growing harmonics and the ones constructed using $1/r^{n+1}$ are the decaying harmonics. Since the $(\theta, \phi)$ dependence is same, we can always convert a decaying harmonic to a growing harmonic by multiplying it with $r^{2n+1}$.

Now we know that the vector solution to the Laplace equation is proportional to $\vect x/r^3$. This is a decaying harmonic. So we multiply it by $r^{2(1)+1} = r^3$ to get the growing harmonic $\vect x$. Therefore the force $\vect F$ will be in general proportional to a linear combination of the two solutions. But we want the field inside the gravitating body, so the the force at the center must not be singular. Therefore $\vect x/r^3$ is rejected and we get $\vect F \propto \vect x $. So $\vect F$ must be linear in $\vect x$.

\subsection{Gravitation pressure}

The gravitational field of a planet pulls it inwards. This causes a pressure which keeps it from shrinking to a point. This pressure can be calculated as follows
\begin{align*}
\grad p &= -\rho_0 \grad \Psi_G\\
\Rightarrow p &= p_0 - \rho_0 \Psi_G \\
\Rightarrow p &= p_0 + 2 \pi \rho_0^2 G (R^2 - r^2/3) \\
\Rightarrow p &= \hat{p_0} - 2 \pi \rho_0^2 G r^2/3
\end{align*}
Pressure at the surface ($r=R$) will be extremely small (only due to the atmosphere) compared to pressure in layers inside the planet. We can assume it to be zero. Therefore, $p|_{r=R} = 0$. Using this we get,
\begin{align*}
\hat{p_0} &= 2 \pi \rho_0^2 G R^2/3 \\
\Rightarrow p &= \frac{2 \pi \rho_0^2 G}{3} (R^2 - r^2)
\end{align*}
This is the pressure which keeps a self-gravitating body from shrinking to a point.

\section{Self-gravitating equilibrium with rotation}
The transformation between two first order tensors must be mediated by a second order tensor. For a sphere, we obtained the relation $\vect F = -\grad \Psi = C \dl_{ij}\vect x$. For a spheroid with axis along $\vect p$, this becomes $\grad \Psi = (C_1 p_ip_j + C_2 \dl_{ij})x_j$. 

For the case of a rotating self-gravitating body, the eccentricity $(e)$ is a function of ratio of rate of rotation and gravity. Therefore, our aim is to find 
\begin{align*}
\frac{\Omega^2 R}{g} &= f(e)
\end{align*}
If we put $g = GM/R^2$ and $M=\rho_0 R^3$, then the equation becomes
\begin{align*}
\frac{\Omega^2}{\rho_0 G} &= f(e)
\end{align*}
This function turns out to be
\begin{align*}
\frac{\Omega^2}{2 \pi \rho_0 G} &= e^2(1-e^2)^{1/2}\int_0^{\infty}\frac{\lambda d\lambda}{(1+\lambda)^2(1-e^2+\lambda)}
\end{align*}

which indicates that for any given rate of rotation, there are two possible equilibrium eccentricities. For small rates of rotation (as for most planets), one of these eccentricities $e \sim 1$ which signifies a nearly flat pancake like body, while the other eccentricity $e \sim 0$ which is indicative of a nearly spherical planet.

Beyond a critical rate of rotation, there do not exist any real roots i.e. centrifugal forces dominate and an equilibrium between centrifugal and gravitational forces is never established.

\section{Appendix}

\subsection{Constant field inside a gouged out sphere}
(figures required)

\subsection{Derivation for Self-gravitating equilibrium with rotation}























\end{document}


