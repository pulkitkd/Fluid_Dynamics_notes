\documentclass[11pt, letterpaper]{article}
\usepackage[utf8]{inputenc}
\usepackage{amsmath}
\usepackage{bm}

\newcommand{\e}{\epsilon}
\newcommand{\dl}{\delta}
\newcommand{\1}{\bm{1}}
\newcommand{\pd}[2]{\frac{\partial #1}{\partial #2}}
\newcommand{\uu}[1]{\underline{\underline{#1}}}
\newcommand{\un}[1]{\underline{#1}}

\title{Lecture 4: Introduction to Cartesian tensors}
\begin{document}
\maketitle

The key ideas are
\begin{itemize}
  \item Index notation for basic operations 
  \item Coordinate transformations for vector and tensors
  \item Isotropic tensors
\end{itemize}

\section{Index notation}
$$\bm{a}\cdot\bm{b}=a_ib_i$$
$$\bm{a}\wedge\bm{b}=\e_{ijk} a_j b_k$$
$$\bm{a}\bm{b}=a_ib_j$$

Using these notations, we can briefly define some quantities of interest in transport phenomena.\\
\begin{itemize}
\item The gradient vector
$$\nabla T = \pd{T}{x_i}$$
which points in the direction of fastest rate of change of the scalar field $T$. Its inner product with a direction gives the rate of change of $T$ in that direction.

\item The divergence of velocity field
$$\nabla \cdot \bm{u} = \pd{u_i}{x_i}$$
which gives the normalized volumetric rate of change of a local volume element ($\frac{1}{V}\pd{V}{t}$). It is zero for an incompressible flow.

\item The curl of velocity field (vorticity)
$$\nabla \wedge \bm{u} = \e_{ijk}\pd{u_k}{x_j} $$
which is a measure of the local rotation rate of the fluid.

\item The velocity gradient tensor
$$\nabla \bm{u} = \pd{u_i}{x_j}$$
which has information about the volumetric change, rotation as well as the local shear rate of the flow. A second order tensor packs much more information than a vector. 
\end{itemize}

As examples of their usage, we can write down few equations in index notation
\begin{itemize}
  \item Convection-diffusion equation
$$\pd{T}{t} + \bm{u}\cdot \nabla T = \kappa \nabla^2 T \hfill \text{\footnotesize{\quad \quad \quad     (Gibbs notation)}}$$
$$\pd{T}{t} + u_j\pd{T}{x_j} = \kappa \pd{^2T}{x_i \partial x_i} \hfill \text{\footnotesize{\quad \quad \quad     (Index notation)}}$$

  \item Navier-Stokes equation
$$\rho(\pd{\bm{u}}{t} + \bm{u}\cdot\nabla\bm{u}) = -\nabla P +\mu \nabla^2 \bm{u} \text{\footnotesize{\quad \quad \quad     (Gibbs notation)}}$$
$$\rho(\pd{u_i}{t} + u_j\cdot\pd{u_i}{x_j}) = -\pd{P}{x_i} +\mu \frac{\partial^2 u_i}{\partial x_j \partial x_j} \text{\footnotesize{\quad \quad \quad     (Index notation)}}$$

\end{itemize}

We will henceforth be denoting second order tensors with a double underbar as $\uu{A}$ and vectors with a single underbar as $\un{A}$. Some identities for products of tensors are given below
\begin{align*}
\uu{B}\cdot \un{a} &= B_{ij} a_j \\
\uu{B} \cdot \uu{C} &= B_{ik}C_{kj} \\
\uu{B} \colon \uu{C} &= B_{ij}C_{ji} \\
\end{align*}

The last of these is used in calculation of viscous dissipation which is given by $2\mu\uu{E}\colon\uu{E}$, where $\uu{E}$ is a second order tensor given by
\begin{align*}
E_{ij} &= \frac{1}{2}(\nabla \un{u} + \nabla \un{u}^T)\\
&=\frac{1}{2}(\pd{u_i}{x_j} + \pd{u_j}{x_i})
\end{align*}

where the form highlights the symmetry of the strain-rate tensor. Two more important relations are
\begin{itemize}
  \item Scalar triple product
$$\un{a}\cdot(\un{b}\wedge\un{c}) = \e_{ijk}a_ib_jc_k$$
which is the determinant of the matrix with $\un{a},\un{b}$ and $\un{c}$ as its columns.

  \item Trace of $\uu{B} = B_{ii}$
\end{itemize}

\section{Coordinate Transformation}

\subsection{Vectors}

We want to find out how does the representation of a vector changes when the coordinate system is transformed into another coordinate system. We will mainly be concered with Orthogonal transformations. Mathematically, this means that, the transformation matrix $L$ is such that
$$L L^T = I \Rightarrow L = L^{-1}$$
Physically, these transformations preserve angles between any two lines. Specifically, if two vectors were orthogonal before the transformation, then their transformed counterparts will also be orthogonal. Orthogonal transformations comprise of rotations and reflections of the coordinate system. The group of all possible rotations of a $n$ dimenisonal coordinate system is known as the $SO_n$ group (Special Orthogonal group in n dimenisons).

Note that, in 3D, if the transformation is a rotation, then the matrix representing such a transformation has one and only one real eigenvalue, which is equal to 1. This is because the rotation matrix modifies every vector in the domain except the vector along its axis which is left unrotated (hence it's an eigenvector) and unstretched (hence the eigenvalue corresponding to it is 1). Other eigenvalues are imaginary. These transformations preserve the length of the vectors and the angle between any two vectors. In 2D, all vectors are transformed and hence both the eigenvalues are imaginary.

Let $L$ be such a tranformation in 3D that acts on the \textit{orthonormal unit vectors} $[\un{1}_1,\un{1}_2,\un{1}_3]$ and yields the transformed \textit{orthonormal unit vectors} $[\un{1'}_1,\un{1'}_2,\un{1'}_3]$. Then, we can write
$$\un{1}'_i  = l_{ij}\un{1}_j$$
The second order tensor $l_{ij}$ has nine elements, but they are not all indepenedent. The constraint of orthonormality of the transformed vectors gives six relations between the elements of $l_{ij}$
\begin{align*}
\un{1}'_i \cdot \un{1}'_j &= 0 \text{\quad \footnotesize{if $i\neq j$ (ensures orthogonality)}}\\
&= 1 \text{\quad \footnotesize{if $i = j$ (ensures normality)}}
\end{align*}


and hence a second order tensor in 3D, that represents an orthogonal transformation has only 3 independent parameters. For rotations, two of these are required to fix the axis and one more to specify the angle of rotation. For reflections, all three are needed to specify the plane of reflection. 


For 2D, we get 3 constraints and hence the angle of rotation is the only parameter. What about 4 and higher dimensions? In a D dimensional space, we will have D constraints due to orthogonality and ${D\choose 2}$ due to normality. This results in $D(D-1)/2$ independent parameters.

Once the transformation rule for the unit vectors is clear, the transformation for any given vector $\un{a} = a_i\un{1}_i$ can be written as

$$\un{a}'_i= l_{ij}\un{a}_j$$
$$\un{a}_j = l_{ji}\un{a}'_i$$

\subsection{Second order tensors}

Let there be a second order tensor $\uu{B}$ with eigenvectors $[\un{X}_1,\un{X}_2,\un{X}_3]$. We assume a symmetric second order tensor which allows us to use the fact that three orthogonal eigenvectors will exist. Then, such a tensor can be written as
$$\uu{B}=[\lambda_1\un{X}_1\un{X}_1+\lambda_2\un{X}_2\un{X}_2+\lambda_3\un{X}_3\un{X}_3]$$

Let the transformed version of $\un{B}$ be $\un{B}'$ with eigenvectors $[\un{X}'_1,\un{X}'_2,\un{X}'_3]$. Then,
$$\uu{B}'=[\lambda_1\un{X}'_1\un{X}'_1+\lambda_2\un{X}'_2\un{X}'_2+\lambda_3\un{X}'_3\un{X}'_3]$$

Now we can use the trasformation rule for vectors $\un{a}'_i= l_{ij}\un{a}_j$ and transform the eigenvectors. This allows us to write down the final transformation rule in Gibbs notation as
$$\uu{B}'=\uu{L}\cdot\uu{B}\cdot\uu{L^T}$$
Or in index notation
\begin{align*}
B'_{ij} &= l_{ik}B_{km}l^T_{mj}\\
\Rightarrow B'_{ij} &= l_{ik}B_{km}l_{jm}\\
\Rightarrow B'_{ij} &= l_{ik}l_{jm}B_{km}
\end{align*}

The final expression is highly sugggestive of a pattern which will allow us to generalize this expression to higher order tensors

$$B'_{i_1i_2i_3...i_n} = l_{i_1j_1}l_{i_2j_2}l_{i_3j_3}...l_{i_nj_n}B_{j_1j_2j_3...j_n} \text{ \quad \quad \footnotesize{For the case of true tensors}}$$



$$B'_{i_1i_2i_3...i_n} = Det(\uu{L})l_{i_1j_1}l_{i_2j_2}l_{i_3j_3}...l_{i_nj_n}B_{j_1j_2j_3...j_n} \text{ \quad \quad \footnotesize{For pseudo tensors}}$$

Pseudo-tensors and pseudo-vectors are tensors which require a convention for their definition (such as the right hand screw rule). Their sign can change based on the convention. They do not directly represent physical quantities. E.g. vorticity, angular velocity and magnetic field. Another way to distinguish pseudo tensors is that they do not change direction upon reflection of the physical apparatus. A ball travelling to the right will be travelling to the left in the mirror; but a disc rotating clockwise is still rotating clockwise in the mirror. Its angular velocity vector doesn't flip sign. 

\subsection{Quotient Rule}
A transformation or a physical law relating a vector to another vector must be mediated via a second order tensor; a law relating two second order tensors be mediated via a fourth order tensor and so on.

A law relating a vector to a second order tensor must be mediated by a third order tensor, so that two of the indices of the second and third order tensors can contract and result in a vector.

\section{Isotropic Tensors}
\begin{itemize}
  \item Zeroth order\\
  All scalars are isotropic except psedo-scalars. Pseudo-scalars form upon contraction of a true vector with a pseudo-vector and their sign changes with convention.

  \item First order (Vectors)\\
  The Null vector is the only isotropic vector. We are looking for a vector that is impervious to any orthogonal transformation. Since every vector suffers a change of direction under some or the other rotation, the only vector that remains unaffected under all rotations is the vector without a direction.

  \item Second order tensors\\
  $\dl_{ij}$ and its scalar multiples are the only second order isotropic vector. Since any vector is an eigenvector of the identity matrix, it implies that there is no special direction for such a matrix. Hence any direction changing transformation such as a rotation will leave it unchanged. 

  \item Third order tensors\\
  There is no true third order isotropic tensor. $\e_{ijk}$ and its scalar multiples give a pseudo third order isotropic tensor.

  \item Third order tensors\\
  All higher order isotropic tensors are constructed using $\e_{ijk}$ and $\dl_{ij}$. E.g.
$$D_{ijkl} = c_1\dl_{ij}\dl_{kl} + c_2\dl_{ik}\dl_{jl} + c_3\dl_{il}\dl_{jk} $$
\end{itemize}

Some examples of the use of isotropy in tensors are given below

\begin{itemize}
  \item Fourier's law \\
  Fourier's law relates heat flux vector $\un{q}$ to the temperature gradient vector $\nabla T$
$$
q_i = K_{ij} \pd{T}{x_j}
$$

Here the thermal conductivity tensor $K_{ij}$ is a second order tensor with 3 independent elements. However, for an isotropic material, the tensor will also be isotropic because it should not prefer any one direction over the other. Therefore, we can write $K_{ij} = \kappa \dl_{ij}$ and there remains only one independent parameter to be determined experimentally.

  \item Newton's law of viscosity \\
  This is a relation between the stress tensor $\tau_{ij}$ and the strain-rate tensor $\nabla \un{u}$. Both of these are second order and hence the relation must be mediated by a fourth order tensor representing viscosity in some sense.
$$
\tau_{ij} = D_{ijkl} \pd{u_l}{x_k}
$$

If viscosity doesn't change with direction in a medium, then $\uu{D}$ must be an isotropic tensor. Hence we can write
$$D_{ijkl} = c_1\dl_{ij}\dl_{kl} + c_2\dl_{ik}\dl_{jl} + c_3\dl_{il}\dl_{jk} $$

But $\tau_{ij}$ is a symmetric tensor. Therefore, terms formed by interchange of $i$ and $j$ must be equal and hence $c_2$ must be equal to $c_3$. This gives us
\begin{align*}
D_{ijkl} &= \bigg(c_1\dl_{ij}\dl_{kl} + c_2(\dl_{ik}\dl_{jl} + \dl_{il}\dl_{jk})\bigg)\pd{u_l}{x_k}\\
D_{ijkl} &= c_1\pd{u_l}{x_l}\dl_{ij} + c_2(\pd{u_i}{x_j}+\pd{u_j}{x_i})
\end{align*}

Here, $c_1$ is associated with the divergence and hence informs us of the resistance of the fluid to change in its volume. This is the bulk viscosity of the fluid.

$c_2$ is associated with the strain-rate and tells about the resistance of the fluid to a shearing deformation (and also volumetric deformation). This leads to shear viscosity.

\end{itemize}



\section*{Appendix}
\subsection{Expanded form for some index notation results}
\begin{align*}
\bm{a} \cdot \bm{b} &= (a_1\1_1+a_2\1_2+a_3\1_3)\cdot(b_1\1_1+b_2\1_2+b_3\1_3)\\
&= \sum_{i=1}^{3} a_i \1_i \cdot \sum_{j=1}^{3} b_j \1_j\\
&= \sum_{i=1}^{3} \sum_{j=1}^{3} a_i b_j \1_i \cdot \1_j\\
&= \sum_{i=1}^{3} a_i b_i\\
&= a_i b_i  
\end{align*}

In the last step, we have used Einstein's summation convention which mandates that repeated indices are to be summed over. Henceforth, repeated indices will imply a summation over that index.

\begin{align*}
\bm{a} \wedge \bm{b} &= (a_1\1_1+a_2\1_2+a_3\1_3)\wedge(b_1\1_1+b_2\1_2+b_3\1_3)\\
&=\sum_{i=1}^{3} a_i \1_i \wedge \sum_{j=1}^{3} b_j \1_j\\
&=\sum_{i=1}^{3}  \sum_{j=1}^{3} a_i b_j \1_i \wedge  \1_j\\
&=\sum_{i=1}^{3}  \sum_{j=1}^{3} a_i b_j \e_{ijk}\1_k\\
&=\e_{ijk}a_ib_j\\
&=\e_{ijk}a_jb_k
\end{align*}

Where we have just relabelled the indices in the last step so that the first index $i$ remains the direction specifying index. 


\subsection{Area under a Gaussian}
The area under a suitably defined Gaussian is independent of its standard deviation. 

\subsection{Gradient vector in cylindrical and spherical coordinates}
In any coordinate system, the gradient vector can be written immidiately if we incorporate the metric factors relevant to the coordinate system. Including the metric factor converts a small change in a coordinate variable to a small change in length along that coordinate (E.g.$d\theta$ to $rd\theta$ ).

\subsection{Proof for vector transformations}
$$\un{a}'_i= l_{ij}\un{a}_j$$
$$\un{a}_j = l_{ji}\un{a}'_i$$

\subsection{Symmetric matrices have orthogonal eigenvectors}

\subsection{Proof for the quotient rule}

\subsection{Proof that $\dl_{ij}$ and $\e_{ijk}$ are isotropic}


\end{document}
